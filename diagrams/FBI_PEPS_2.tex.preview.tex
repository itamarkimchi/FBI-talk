%& C:\Users\Brayden\AppData\Roaming\TikzEdt\TikzEdt\023~1.0\TEMP_H~1
\begin{document}
% y=\sqrt{3/4}*(minimum size)/2
%

%http://tex.stackexchange.com/questions/6019/drawing-hexagons
\begin{tikzpicture}[x=11.25mm,y=6.5mm]
\begin{scope}[decoration={
    markings,
    mark=at position 0.5 with {\arrow{>}}}
    ]
\clip (-2, -2.52) rectangle (4, 3.56);

  \tikzset{box/.style={
      regular polygon,
      regular polygon sides=6,
      minimum size=15mm,
      inner sep=0mm,
      outer sep=0mm,
      rotate=0,
     % dotted,
      thin,
      black,
      fill=white,
      draw}}
 
      \tikzset{
    abox/.style={
      regular polygon,
      regular polygon sides=6,
      minimum size=9mm,
      inner sep=0mm,
      outer sep=0mm,
      rotate=0,
     dotted,
     thin,
      white,
      draw
    }
    } 
      
 \tikzset{wb/.style={
 	regular polygon,
	regular polygon sides=6,
	minimum size=3mm,
      inner sep=0mm,
      outer sep=0mm,
      rotate=0,
      % dotted,
      very thick,
      blue,
      fill=blue!30,
      draw}}
      
\tikzset{operator/.style={
	circle=1pt,
	draw=orange!100, 
	very thick,
	fill=orange!80,
	inner sep=2pt}}

\tikzset{smalldot/.style={circle=1pt,draw=black!100,fill=black!100,inner sep=1pt}}
\tikzset{smallwdot/.style={circle=1pt, very thick, draw=blue!100,fill=blue!30,inner sep=1pt}}

\foreach \i in {-2,...,2}{
	\foreach \j in {-2, ...,  2} {
		\pgfmathtruncatemacro{\x}{2*\i+1}
		\pgfmathtruncatemacro{\y}{2*\j+1}
             \node[box] (H\x\y) at (\x,\y) {};
             \node[abox](Q\x\y) at (\x,\y){};
           \foreach \k in {1,...,6}{
           		\node[smalldot] (A\x\y\k) at (Q\x\y.corner \k) {};
           		 \draw[thick, postaction={decorate}] (A\x\y\k) -- (H\x\y.corner \k);
              	\node[operator] (D\x\y\k) at (H\x\y.corner \k) {};
              	 \node[black] at (H\x\y.corner \k){\tiny D};
              	}
            \foreach \k in {1,2,3,5}{
		\pgfmathtruncatemacro{\z}{\k+1}
            \draw[thick, postaction={decorate}] (A\x\y\k) -- (A\x\y\z);
             }
             \draw[thick, postaction={decorate}] (A\x\y6) -- (A\x\y1);
              \node[smallwdot] (B\x\y) at (\x,\y-0.1) {\tiny 1};
 		\draw[thick, postaction={decorate}]  (B\x\y)--(A\x\y5);
        }
}

\foreach \i in {-2,...,2}{
	\foreach \j in {-2,...,2} {
		\pgfmathtruncatemacro{\x}{2*\i}
		\pgfmathtruncatemacro{\y}{2*\j}
             \node[box] (H\x\y) at (\x,\y) {};
             \node[abox](Q\x\y) at (\x,\y){};

              \foreach \k in {1,...,6}{
              	\node[smalldot] (A\x\y\k) at (Q\x\y.corner \k) {};
              	 \draw[thick, postaction={decorate}] (A\x\y\k) -- (H\x\y.corner \k);
              	\node[operator] (D\x\y\k) at (H\x\y.corner \k) {};
              	\node[black] at (H\x\y.corner \k){\tiny D};
             	}
             	\foreach \k in {1,2,3,5}{
		\pgfmathtruncatemacro{\z}{\k+1}
            \draw[thick, postaction={decorate}] (A\x\y\k) -- (A\x\y\z);
             }
             
             \draw[thick, postaction={decorate}] (A\x\y6) -- (A\x\y1);
              \draw[thick, postaction={decorate}] (A\x\y6) -- (A\x\y1);
              \node[smallwdot] (B\x\y) at (\x,\y-0.1) {\tiny 1};
 		\draw[thick, postaction={decorate}]  (B\x\y)--(A\x\y5);
        }
}


 
 \end{scope}

%\draw[thick, black] (-2, -2.4) rectangle (4, 3.6);


\usetikzlibrary{calc}
\pgftransformreset
\node[inner sep=0pt,outer sep=0pt,minimum size=0pt,line width=0pt,text width=0pt,text height=0pt] at (current bounding box) {};
%add border to avoid cropping by pdflibnet
\foreach \border in {0.1}
  \useasboundingbox (current bounding box.south west)+(-\border,-\border) rectangle (current bounding box.north east)+(\border,\border);
\newwrite\metadatafile
\immediate\openout\metadatafile=\jobname_BB.txt
\path
  let
    \p1=(current bounding box.south west),
    \p2=(current bounding box.north east)
  in
  node[inner sep=0pt,outer sep=0pt,minimum size=0pt,line width=0pt,text width=0pt,text height=0pt,draw=white] at (current bounding box) {
\immediate\write\metadatafile{\p1,\p2}
};
\immediate\closeout\metadatafile
\end{tikzpicture}

\end{document}
