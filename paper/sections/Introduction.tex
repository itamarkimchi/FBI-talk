%!TEX root = ../fbi.tex

\section{Introduction}

\bela{This is pretty clich\'{e} right now. It's also missing all references.}

Recent years have seen the proliferation of phases that defy a description
within the well-known framework of local order parameters. These phases are
not characterized by a local order parameter, but rather by more subtle,
non-local features. An important class of such phases are topological phases,
with key characteristics such as exotic, anyonic excitations. Another
important class are symmetry-protected topological phases, which have
garnered a great deal of attention since their original discovery.

Famously, the Lieb-Schults-Mattis (LSM) theorem and its extensions gives a set of
conditions under which such an exotic phase not only can, but indeed \emph{must}
exist. In particular, they forbid the existence of a featureless state
-- a state that neither spontaneously breaks a symmetry, nor displays topological
order, nor has power-law correlations and is thus ''gapless'' -- in systems
with a fractional filling per unit cell. Extensions to LSM in \onlinecite{parameswaran2013-2} further
forbid some site fractional site fillings for non-symmorphic lattices.

In Ref.~\onlinecite{kimchi2013}, an interesting special case was considered,
namely that of half-filled bosons on the honeycomb lattice. In this case, where
there is one boson per unit cell of two sites, the LSM allows the existence
of a featureless state. At the same time, however, symmetry guarantees that
a free-fermion spectrum is gapless at certain high-symmetry points and there is
thus no localized Wannier basis. This implies that
a featureless state cannot be constructed by filling a permanent
of localized Wannier orbitals, and any construction must thus involve interactions
in a non-trivial way. The explicit construction of Ref.~\onlinecite{kimchi2013},
henceforth dubbed honeycomb featureless boson insulator (HFBI),
will be discussed in detail in Section~\ref{sec:fbi}.
\bela{Possibly move the first few paragraphs of Sec II, or at least a short summary, here?}

Beyond showing that the construction gives a featureless state, we also want 
to argue that it lies in a distinct SPT phase, and show how to separate this 
phase from classical insulators and other featureless states. Instead of local 
order parameters, SPT phases are distinguished by features that appear at 
phase boundaries, such as a gapless edge spectrum, and by entanglement 
properties of the groundstate wavefunction, such as a gapless \em{entanglement 
edge spectrum} or a topological invariant. These properties have been previously used to identify SPTs with on-site symmetries, such as free fermion or interacting topological insulators, as well as free fermion crystalline topological insulators. \brayden{citation needed}

The main focus of this paper will be on the edge properties of the HFBI.
In particular, we will find that while the state is featureless in the bulk, it
features a gapless edge state which is protected by a combination of lattice inversion
and spin symmetry. We will argue for this by calculating the entanglement spectrum and showing
that the low-lying spectrum is to great accuracy described by that of a $c=1$
conformal field theory; in addition, we will show that there is exact double 
degeneracy throughout the entanglement spectrum for certain geometries.

A simpler way to classify an SPT phase beyond edge spectra is using 
topological invariants, which are numbers that can be computed from the bulk 
wavefunction that provably must be constant throughout the phase. These 
invariants measure how the action of the symmetry is implemented on the 
physical edge states of open systems or on the Schmidt states of an 
entanglement decomposition. These invariants feature heavily in the 
classification of SPT phases with on-site symmetry, and similar invariants 
that apply to free-fermion states have been used for topological crystaline 
insulators. By contrast, topological invariants for interacting states 
protected by lattice symmetries in more than one dimension aren't well 
understood. We will discuss the action of the symmetry on the edge of the HFBI 
state and progress towards the goal of finding a topological invariant to 
identify the corresponding phase.

All these calculations are performed in a framework of tensor network states.
The simplest example of a tensor network for a 1D system is a matrix
product state; translationally-invariant matrix product states (MPS)
are represented by a single rank 3 tensor $A_p^{ij}$ which specifies
the wavefunction coefficients in a given basis local basis as a
product of matrices $$\ket{\psi} = \sum\limits_{\{p_i\}} A_{p_1}
A_{p_2} ... A_{p_N} \ket{p_0 p_1 ... p_N},$$ with one matrix for each
site in the one-dimensional system.

PEPS are the generalization of MPS to two and higher dimensions, where
each site in the system is represented by a rank $z+1$ tensor, where
$z$ is the coordination number of the lattice, and wavefunction
coefficients are similarly given by contracting over all virtual
indices as shown in Figure \ref{fig:PEPS}.\cite{verstraete2004}.

The advantage of a PEPS representation for SPT identification is that the 
Schmidt decomposition can be described using the Hilbert space of the virtual 
legs crossing the cut instead of the physical Hilbert space on either side of 
the cut.