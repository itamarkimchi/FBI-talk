%!TEX root = ../fbi.tex

\section{Introduction}

\bela{Things that need to be mentioned in here somewhere:
More on classification of SPTs, limitations thereof for crystal SPTs, and topological invariants.}

The discovery of time-reversal invariant band insulators that are not adiabatically connected to the atomic limit in 2007~\cite{...}
has spurred the discovery of a broad array of phases where symmetries protect subtle, non-local features that distinguish
them from trivial, unentangled insulators. These phases, coined symmetry-protected topological (SPT) phases~\cite{...}, have
by now been observed in several experimental realizations both in two and three dimensions~\cite{somereview} and an extensive
mathematical framework has been developed for their characterization and classification~\cite{wen...}.
One of the key distinguishing features of these phases is their edge physics, which often exhibits gapless features such as
the helical edge states of the quantum spin Hall effect~\cite{...} and the protected Dirac cones on the surface of three-dimensional
$\mathbb{Z}_2$ topological insulators~\cite{...}.
\bela{Segue into entanglement.}

Here, we study a bosonic insulator on the honeycomb lattice that was first proposed by Kimchi {\it et al.}~\cite{kimchi2013}.
This state fills an interesting gap in the Lieb-Schultz-Matthis (LSM) theorem~\cite{...}:
While the LSM theorem forbids the existence of a featureless state
-- a state that neither spontaneously breaks a symmetry, nor displays topological
order, nor has power-law correlations and is thus ''gapless'' -- in systems
with a fractional filling per unit cell, such a state is in principle allowed on the honeycomb lattice
at half filling per site, and thus unit filling per unit cell. However, symmetry guarantees that
a free-fermion spectrum is gapless at certain high-symmetry points and there is
thus no localized Wannier basis.
This implies that a featureless state on the honeycomb lattice cannot be constructed by filling a permanent of localized Wannier
orbitals, and any construction must thus involve interactions in a non-trivial way. Ref.~\onlinecite{kimchi2013} pursued
an approach of constructing a permanent wavefunction by filling local and symmetric
orbitals that are not orthogonal and
it was argued that that the resulting wavefunction is indeed featureless.
In particular, using numerical simulations it was found that the state exhibits
isotropic and exponentially decaying correlations, and arguments were presented that it is not
topologically ordered.

While we confirm the featureless bulk of the state, we show that nevertheless the entanglement of the state cannot be removed while preserving
all symmetries, i.e. it constitutes a symmetry-protected phase. The relevant symmetry is a combination of charge
conservation and lattice symmetries, which together protect universal features in the entanglement spectrum. In
particular, we show that the low-lying entanglement spectrum is to great accuracy described by that of a $c=1$
conformal field theory, and that there is an exact double 
degeneracy throughout the entanglement spectrum for certain geometries, which is protected
by the symmetries of the state and thus serves as a topological invariant identifying the SPT order.
Since lattice symmetry is involved crucially, this provides one of the first examples for an SPT of interacting bosons
protected by lattice rather than on-site symmetries.

%A simpler way to classify an SPT phase beyond edge spectra is by using 
%topological invariants, which are numbers that can be computed from the bulk 
%wavefunction that provably must be constant throughout the phase. These 
%invariants measure how the action of the symmetry is implemented on the 
%physical edge states of open systems or on the Schmidt states of an 
%entanglement decomposition. These invariants feature heavily in the 
%classification of SPT phases with on-site symmetry, and similar invariants 
%that apply to free-fermion states have been used for topological crystaline 
%insulators. By contrast, topological invariants for interacting states 
%protected by lattice symmetries in more than one dimension aren't well 
%understood. We will discuss the action of the symmetry on the edge of the HFBI 
%state and progress towards the goal of finding a topological invariant to 
%identify the corresponding phase.

All of these properties of the phase become accessible through a description of the state
as a projected entangled-pair state (PEPS)~\cite{verstraete2004}. These states form a specific class of tensor
network states that corresponds to a generalization of the well-known matrix-product state (MPS)
framework to higher dimensions. PEPS have been shown to be a powerful description of many
classes of gapped systems, including topologically ordered and SPT phases. Here, we have an
exact description of the state as a PEPS, allowing us to extract properties such as the entanglement
spectrum and the topological invariants exactly on certain geometries; we emphasize that these
properties of the state are not accessible to other numerical methods.

