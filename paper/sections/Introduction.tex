%!TEX root = ../fbi.tex

\section{Introduction}


Featureless insulator problem description from   \onlinecite{parameswaran2013}
\bi 
\item Classical Mott insulators have integer site filling
\item LSM forbids featureless insulators at fractional unit cell 
filling. Site-filling can still be multiples of $1/n$
\item Extensions to LSM in \onlinecite{parameswaran2013-2} further 
forbid some site fractional site fillings for non-symmorphic lattices.
\ei 

In addition to finding featureless states, we also want to determine
 if different featureless states are in the same phase, i.e. whether 
 they can be connected without a phase transtion, but while preserving 
 the symmetry of the Hamiltonian H. Of particular interest are those 
 featureless states that cannot be connected adiabatically to a 
 product state without breaking the symmetry; these are called 
 symmetry protected topological (SPT) phases. Instead of local order 
 parameters, the distinguishing "features" of these phases appear at 
 phase boundaries, such as the edge of an open system, and in 
 entanglement properties of the bulk wavefunction.



 connecting path of Hamiltonians with featureless ground states everywhere along the path. Since the ground state degeneracy on a torus changes

SPT phases for a symmetry group G are featureless states 

The simplest example of a tensor network for a 1D system is a matrix 
product state; translationally-invariant matrix product states (MPS) 
are represented by a single rank 3 tensor $A_p^{ij}$ which specifies 
the wavefunction coefficients in a given basis local basis as a 
product of matrices $$\ket{\psi} = \sum\limits_{\{p_i\}} A_{p_1} 
A_{p_2} ... A_{p_N} \ket{p_0 p_1 ... p_N},$$ with one matrix for each 
site in the one-dimensional system.

PEPS are the generalization of MPS to two and higher dimensions, where 
each site in the system is represented by a rank $z+1$ tensor, where 
$z$ is the coordination number of the lattice, and wavefunction 
coefficients are similarly given by contracting over all virtual 
indices as shown in Figure \ref{fig:PEPS}.\cite{verstraete2004}