\section{Symmetry protection}
\label{sec:symmetry}

Notice that many points in the entanglement spectrum are doubly degenerate - those that are
assigned a non-zero $U(1)$ charge - and on odd circumference cylinders, the
entire entanglement spectrum is doubly degenerate. In this section, we will discuss how
the corresponding degenerate Schmidt states are related by a symmetry of the HFBI wavefunction. 
As discussed in Ref.~\onlinecite{pollmann2010} and reviewed in the Appendix~\ref{Appendix:MPS},
this symmetry action can be used to diagnose one-dimensional symmetry protected topological order.
for which degeneracy throughout the entire entanglement spectrum is a robust feature.
We will demonstrate that the odd circumference cylinders, considered as one-dimensional states, 
are indeed SPTs protected by a combination of lattice inversion and charge parity symmetries.

\brayden{Connection between 1D SPT and 2D SPT results: Maybe take this out or put this later?}

The degenerate entanglement spectra on odd circumference cylinders should be a feature 
robust to small perturbations of any parent Hamiltonian for this state, as long as the 
perturbations respect the above symmetry, and as long as perturbations respect the translational
symmetry of the lattice - so that the notion of 'odd circumference cylinder' is well defined.
So the 1D SPT result can be extended to a 2D SPT result, as long as we include translational
symmetry in the protecting group. It may be that translational symmetry isn't actually necessary
to argue for this protection, but additional evidence is needed to argue for that. 

%

%However, the entanglement spectrum of this one wavefunction is not enough to
%determine the SPT nature of the phase. First, it isn't clear what group or
%groups of symmetries can be used to protect the gapless edge. It could be that
%the entire symmetry group - $U(1)$ boson number conservation, \brayden{time
%reversal symmetry?}, as well as the entire rotational and translational group
%of the lattice - must be preserved to protect the edge. Or, as we will argue
%is the case, a much smaller subgroup could be used to protect the edge. This
%leads to a wider class of perturbations that leave the edge intact - and since
%we will argue that translation is not needed for protection, the gapless edge should additonally be robust to some types of weak disorder and can be seen in
%small systems with symmetry preserving boundary conditions. Second, the
%entanglement spectrum alone fails to distinguish between other SPT phases with
%the same protecting group - for that, we need a topological invariant. Third,
%we would like to confirm the SPT nature of the phase by perturbing a parent
%Hamiltonian with terms that break various combinations of symmetries and
%seeing if they destroy the gapless edge.

%To determine the protecting group, we can proceed in
%two ways: we could perturb a parent Hamiltonian with terms that break various
%combinations of symmetries and see which perturbations destroy the gapless
%edge, or we can look for a topological invariant by analyzing the action of
%the symmetry of the entanglement edge. We will take up the later question
%first, then return to the prospect of perturbing a parent Hamiltonian in
%Section \ref{sec:perturbations}.


%Given that only odd circumference cylinders are SPTs, one might wonder whether
%odd and even $L$ cylinders approach two differt phases in the thermodynamic
%limit. In Section~\ref{sec:CFT}, we will provide evidence against that
%possibility, by showing that in both cases, the low-energy, linear dispersing
%part of the entanglement spectra can be described by the same conformal field
%theory. Thus, in the thermodynamic limit, the two edge spectra approach the
%same set of points.



