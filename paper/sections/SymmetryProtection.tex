\section{Symmetry-protection}
\label{sec:symmetry}


%However, the entanglement spectrum of this one wavefunction is not enough to
%determine the SPT nature of the phase. First, it isn't clear what group or
%groups of symmetries can be used to protect the gapless edge. It could be that
%the entire symmetry group - $U(1)$ boson number conservation, \brayden{time
%reversal symmetry?}, as well as the entire rotational and translational group
%of the lattice - must be preserved to protect the edge. Or, as we will argue
%is the case, a much smaller subgroup could be used to protect the edge. This
%leads to a wider class of perturbations that leave the edge intact - and since
%we will argue that translation is not needed for protection, the gapless edge should additonally be robust to some types of weak disorder and can be seen in
%small systems with symmetry preserving boundary conditions. Second, the
%entanglement spectrum alone fails to distinguish between other SPT phases with
%the same protecting group - for that, we need a topological invariant. Third,
%we would like to confirm the SPT nature of the phase by perturbing a parent
%Hamiltonian with terms that break various combinations of symmetries and
%seeing if they destroy the gapless edge.

%To determine the protecting group, we can proceed in
%two ways: we could perturb a parent Hamiltonian with terms that break various
%combinations of symmetries and see which perturbations destroy the gapless
%edge, or we can look for a topological invariant by analyzing the action of
%the symmetry of the entanglement edge. We will take up the later question
%first, then return to the prospect of perturbing a parent Hamiltonian in
%Section \ref{sec:perturbations}.

Notice that many points in the spectrum are doubly degenerate - those that are
assigned a non-zero $U(1)$ charge - and on odd circumference cylinders, the
entire entanglement spectrum is doubly degenerate, a property shared with 1D SPTs. This suggests that the the HFBI state as a 1D state with any
fixed odd cylinder circumference $L$ is a 1D SPT. We will now discuss
how to explain the degeneracies in these spectra
using the action of the symmetry on the edge, and how to prove that the odd
circumference cylinder states are indeed 1D SPTs while the even circumference
cylinder states are not. This will shed light on what the appropriate symmetry
group to use.

Given that only odd circumference cylinders are SPTs, one might wonder whether
odd and even $L$ cylinders approach two differt phases in the thermodynamic
limit. In Section~\ref{sec:CFT}, we will provide evidence against that
possibility, by showing that in both cases, the low-energy, linear dispersing
part of the entanglement spectra can be described by the same conformal field
theory. Thus, in the thermodynamic limit, the two edge spectra approach the
same set of points.



%By combining $N$ such blocks together and connecting them with periodic
%boundary conditions, we recover the fact that
%$U_g \ket{\psi} = e^{i \Theta_g} \ket{\psi}$,
%with $\Theta_g = N \theta_g$, since all of the $V_g$ and $V_g^{\dagger}$s
%cancel pairwise.
%$\theta_g$ measures the charge per unit cell



