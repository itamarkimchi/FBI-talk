\begin{frame}[c]{Can we use `band bosons`?}
\vskip-1.5cm
\only<1>{
Band fermions have atomic insulator picture using Wannier functions.

$$W^{\alpha}_R(x) = \int_{B.Z.} d\kk e^{-iR\cdot \kk} \psi^{\alpha}_{\kk}(x)$$

$$d_R^{\alpha \dagger} = \sum\limits_x W^{\alpha}_R(x)c_x^{\dagger}$$ 

$$\ket{\psi} = \prod\limits_R d_R^{\alpha \dagger} \ket{\mathbf{0}} $$

Wannier functions are not unique and often don't respect lattice symmetries depending on choice of phase for original basis functions. 
\\\vspace{1em}
But the resulting 'Slater determinant' wavefunction is symmetric regardless. 
%due to antisymmetrization
%individual wannier functions are not eigenstates unless the band is perfectly flat.
%This picture breaks down when the chern number of a band is not zero, or if bands touch.
}
% A band boson is defined in analogy to this slater determinant wave
\only<2>{
\begin{block}{`Band bosons' a.k.a. Boson Permanent}
{A boson permanent wavefunction created from filling an orbital $\phi_{R+x}$ for each unit cell $R$ with a boson}
\end{block}
Analogous to the Slater determinant for fermions except:
%with a few key differences:
\bi 
\item $\ket{\psi}$ respects lattice symmetries only if $\phi$ does
%Boson permanent wavefunction won't respect lattice symmetries unless orbitals filled do.
\item $H$ needs repulsive interactions to stop Bose condensation
\bi 
\item e.g. $H_{BH}$ with $$d^{\dagger}_R = \sum\limits_i \phi_i b^{\dagger}_i$$
\ei
%use bose-hubbard repulsion
\item Orbitals will need to be localized and orthogonal for $H_{BH}$ to be a parent Hamiltonian
\ei 

%If we could find symmetric, orthogonal, localized orbitals, we can try fill them with bosons to make a 'band' boson featureless insulator with a nice parent Hamiltonian.
%Say by analyzing the fermion tightbinding model 
%Can do this on Kagome but not on honeycomb: the symmetry protection of the band touching is an obstruction
}
\end{frame}