% y=\sqrt{3/4}*(minimum size)/2
%

%http://tex.stackexchange.com/questions/6019/drawing-hexagons
\begin{tikzpicture}[x=7.5mm,y=4.33mm]

\clip (-3, -2.4) rectangle (3, 3.6);

  \tikzset{box/.style={
      regular polygon,
      regular polygon sides=6,
      minimum size=10mm,
      inner sep=0mm,
      outer sep=0mm,
      rotate=0,
      very thick,
      black,
      fill=white,
      draw}}
      
\tikzset{operatorA/.style={
	circle=1pt,
	draw=blue!30, 
	very thick,
	fill=blue!30,
	inner sep=2pt}}

\tikzset{operatorB/.style={
	circle=1pt,
	draw=green!30, 
	very thick,
	fill=green!30,
	inner sep=2pt}}


\foreach \i in {-2,...,2}{
	\foreach \j in {-2, ...,  2} {
		\pgfmathtruncatemacro{\x}{2*\i+1}
		\pgfmathtruncatemacro{\y}{2*\j+1}
	       \pgfmathtruncatemacro{\z}{2*\i}
		\pgfmathtruncatemacro{\w}{2*\j}
             \node[box] (H\x\y) at (\x,\y) {};
             \node[box] (H\z\w) at (\z,\w) {};
        }
}
\foreach \i in {-2,...,2}{
	\foreach \j in {-2, ...,  2} {
		\pgfmathtruncatemacro{\x}{2*\i+1}
		\pgfmathtruncatemacro{\y}{2*\j+1}
       	\foreach \k in {1, 3, 5}{
              	\node[operatorA] (D\x\y\k) at (H\x\y.corner \k) {};
              	 \node[black] at (H\x\y.corner \k){\tiny B};
              	}
            \foreach \k in {2,4,6}{
             	\node[operatorA] (D\x\y\k) at (H\x\y.corner \k) {};
              	 \node[black] at (H\x\y.corner \k){\tiny A};
              	}
 }
 }


\draw[very thick, red] (0, -3) -- (0, 4);
\draw[thick, dotted, black] (-3, -2.4) -- (3, -2.4);
\draw[thick, dotted, black] (-3, 3.6) -- (3, 3.6);

\end{tikzpicture}