\begin{frame}{What is a matrix product state?}
\vskip-1.5cm
Matrix product states provide a parameterization of the space of wavefunctions of a 1D or quasi-1D system.
\only<1, 3->{
\begin{figure}[h]
    \centering
    \scalebox{1}{
    \begin{tikzpicture}[node distance=0.4cm]
\tikzset{gamma/.style={circle=2pt,draw=black!100, very thick, fill=blue!40, inner sep=3pt}}


\begin{scope}[decoration={
    markings,
    mark=at position 0.5 with {\arrow{>}}}
    ]

\foreach \i in {1,...,6} {
	\node[gamma] (G\i) at (1.5*\i,0) {$A$};
	\node (l\i) at ($ (G\i) + (0, 1) $) {$$};
	\draw[very thick, red!100, postaction={decorate}] (G\i) -- (l\i);
	%\node[below of=G\i] {$A$};
}

\foreach \i in {7} {
	\node[] (G\i) at (1.5*\i,0) {$...$};
    }  
    
\foreach \i / \j in {1/2,2/3,3/4,4/5,5/6, 6/7} {
	\draw[very thick, postaction={decorate}] (G\j) -- (G\i);
}
\node (b0) at (0, 0) {$...$};
\draw[very thick, postaction={decorate}] (G1) -- (b0);

\end{scope}
\end{tikzpicture}

    }
\end{figure}
\only<1>{
$$
\ket{\psi^{b_0 b_1}} = \sum\limits_{p_1 ... p_5}   \vbra{b_0} A^{p_1}_1 ... A^{p_5}_5 \vket{b_1}\ket{p_1 ... p_5} 
$$
}
Coefficients of the wavefunction are calculated via a product of matrices, one per site. The matrix at each site depends on the physical state at that site.
}

\only<2>{
\begin{figure}[h]
    \centering
    \scalebox{1}{
    \begin{tikzpicture}[node distance=0.4cm]
\begin{scope}[decoration={
    markings,
    mark=at position 0.5 with {\arrow{>}}}
    ]
\foreach \i in {1,...,5} {
	\node[gamma] (G\i) at (1.5*\i,0) {};
	\node (l\i) at ($ (G\i) + (0, 1) $) {$p_\i$};
	\draw[thick] (G\i) -- (l\i);
	\node[below of=G\i] {$A^\i$};
}
\foreach \i / \j in {1/2,2/3,3/4,4/5} {
	\draw[thick, postaction={decorate}] (G\j) -- (G\i);
}
\node[side] (B) at (0.5, 0) {$B$};
%\node (b1) at (0, 0) {$b_1$};
\draw[thick, postaction={decorate}] (G1) -- node[below]{$b_0$}(B);
\draw[thick, postaction={decorate}] (8.1, 0)--(G5.east);
\draw[thick, postaction={decorate}] (B.west) -- node[below]{$b_1$}(-0.1, 0);
\end{scope}
\draw[thick, dashed] (0.2,0) .. controls (-1,0) and (-0.6,0.6) .. (4,0.5) .. controls (8.5,0.5) and (9,0) .. (7.5,0);
\end{tikzpicture}
    }
\end{figure}

\only<2>{
$$
\ket{\psi} = \sum\limits_{p_1 ... p_5}   Tr(B A^{p_1}_1 ... A^{p_5}_5)\ket{p_1 ... p_5} 
$$
}

}
\only<3->{
Every state has a matrix product state representation formed through the process of repeated SVD.

\only<3>{
\begin{figure}[h]
    \centering
    \scalebox{1.1}{
    \begin{tikzpicture}[node distance=0.6cm]

\node[side, minimum width=1.6cm, minimum height=0.8cm] (psi) at (0,0) {$\psi$};
\node[above=of psi.north west, anchor=south west] (p1) {};
\node[above=of psi](p2) {};
\node[above=of psi.north east, anchor=south east] (p3) {};
\node at (1.3, 0) {=}; 
\node[cside] (psi1) at (2.2,0) {$U_1$};
\node[lambda] (S) at (3.4, 0) {};
\node[cside] (psiR) at (4.8,0) {$V_1$};
\node[above=of psi1.north west, anchor=south west] (pL) {};
\node[above=of psiR.north west, anchor=south west] (q2) {};
\node[above=of psiR.north east, anchor=south east] (qR) {};

\begin{scope}[decoration={
    markings,
    mark=at position 0.5 with {\arrow{>}}}
    ]
 \draw[->] (psi.north -| p1) -- node[left] {$p_1$} (p1);
 \draw[->] (psi) --  node[left] {$p_2$} (p2);
 \draw[->] (psi.north -| p3) -- node[left] {$p_3$} (p3);
 \draw[->] (psi1.north -| pL) -- node[left] {$p_1$} (pL);
 \draw[->] (psiR.north-|q2)-- node[left] {$p_2$} (q2);
 \draw[->] (psiR.north -| qR) -- node[right] {$p_3$} (qR);
 \draw[-, thick, postaction={decorate}]  (S) -- (psi1);
 \draw[-, thick, postaction={decorate}] (S) -- (psiR);
 \end{scope}
 \end{tikzpicture}
    }
    \note{Maximum bond dimension d}
\end{figure}
}
\only<4>{
\begin{figure}[h]
    \centering
    \scalebox{1.1}{
    \begin{tikzpicture}[node distance=0.6cm]
    \node[side, minimum width=1.5cm, minimum height=0.8cm] (psi) at (0,0) {$\psi$};
    \node[above=of psi.north west, anchor=south west] (p1) {};
    \node[above=of psi](p2) {};
    \node[above=of psi.north east, anchor=south east] (p3) {};
    \node at (1.3, 0) {=}; 
    \node[cside] (psi1) at (2,0) {$U_1$};
    %\node[lambda] (S) at (3, 0) {};
    \node[side,minimum width=1cm, minimum height=0.8cm] (psiR) at (4.2,0) {$\psi'$};
    \node[above=of psi1.north west, anchor=south west] (pL) {};
    \node[above=of psiR.north west, anchor=south west] (q2) {};
    \node[above=of psiR.north east, anchor=south east] (qR) {};
\begin{scope}[decoration={
    markings,
    mark=at position 0.5 with {\arrow{>}}}
    ]
    \draw[->] (psi.north -| p1) -- node[left] {$p_1$} (p1);
    \draw[->] (psi) --  node[left] {$p_2$} (p2);
    \draw[->] (psi.north -| p3) -- node[left] {$p_3$} (p3);
    \draw[->] (psi1.north -| pL) -- node[left] {$p_1$} (pL);
    \draw[->] (psiR.north-|q2)-- node[left] {$p_2$} (q2);
    \draw[->] (psiR.north -| qR) -- node[right] {$p_3$} (qR);
    \draw[-, thick, postaction={decorate}] (psiR) -- node[above]{$i_1$}(psi1);
  \end{scope}
\end{tikzpicture}
    }
    \note{Maximum bond dimension d^2}
\end{figure}
}
\only<5>{
\begin{figure}[h]
    \centering
    \scalebox{1.1}{
    \begin{tikzpicture}[node distance=0.6cm]
    \node[side, minimum width=1.5cm, minimum height=0.8cm] (psi) at (0,0) {$\psi$};
    \node[above=of psi.north west, anchor=south west] (p1) {};
    \node[above=of psi](p2) {};
    \node[above=of psi.north east, anchor=south east] (p3) {};
    \node at (1.3, 0) {=};
    \node[cside] (B1) at (2,0) {$U_1$};
    \node[cside] (B2) at (3.5,0) {$U_2$};
    \node[side, minimum width=0.5cm, minimum height=0.8cm](psiR) at (5, 0){$\psi''$};
    %\node[lambda] (S) at (3, 0) {};
    %\node[side,minimum width=1cm, minimum height=0.8cm] (psiR) at (4.2,0) {$\psi'$};
    \node[above=of B1.north, anchor=south] (pL) {};
    \node[above=of B2] (q2) {};
    %\node[above=of psiR.north west, anchor=south west] (q2) {};
    \node[above=of psiR.north, anchor=south] (qR) {};
\begin{scope}[decoration={
    markings,
    mark=at position 0.5 with {\arrow{>}}}
    ]
    \draw[->] (psi.north -| p1) -- node[left] {$p_1$} (p1);
    \draw[->] (psi) --  node[left] {$p_2$} (p2);
    \draw[->] (psi.north -| p3) -- node[left] {$p_3$} (p3);
    \draw[->] (B1.north -| pL) -- node[left] {$p_1$} (pL);
    \draw[->] (B2.north-|q2)-- node[left] {$p_2$} (q2);
    \draw[->] (psiR.north -| qR) -- node[right] {$p_3$} (qR);
    \draw[-, thick, postaction={decorate}] (B2) -- node[above]{$i_1$}(B1);
    \draw[-, thick, postaction={decorate}] (psiR) -- node[above]{$i_2$}(B2);
 %\draw[-] (psiR) -- (S);
  \end{scope}
\end{tikzpicture}
    }
\end{figure}
}
}

\end{frame}