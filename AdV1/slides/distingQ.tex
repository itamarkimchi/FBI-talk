\begin{frame}{Motivating Questions}
\vskip -1.5cm

Are there general principles for distinguishing potential featureless insulator ground states of Hamiltonians?

The theory of {\em symmetry protected topological phases} (SPTs) is a general framework for distinguishing different featureless insulators.

\bi 
\item {\em Topological} - some discrete invariant that won't change under continuous (adiabatic) changes in Hamiltonian
\note{Can change when gap closes}
\item Invariants should be defined for interacting systems that obey certain symmetries 
\item Often features edge fractionalization and degeneracy in open boundary conditions
\item In 1D, universally distinguished by entanglement spectra
\ei

Are there additional constraints on the existence of featureless insulators in interacting systems?

\note{Any such constraint would be an extension of LSM}
\note{Those here who study topologically ordered states could care about this question because any such constraint lowers the burden of proof for T.O (only need to show gapped and symmetric.) (Of course, its still very hard to rule out symmetry breaking under complicated local order parameters or very large unit cells.)}
\note{Is this question at all related to the first question?}

\end{frame}