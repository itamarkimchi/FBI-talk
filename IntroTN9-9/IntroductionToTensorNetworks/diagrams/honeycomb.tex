% http://tex.stackexchange.com/questions/21923/hexagonal-diagrams

\begin{tikzpicture}
%%%  define vertices with coordinates
\coordinate (0;0) at (0,0); 
\foreach \c in {1,...,3}{%  
\foreach \i in {0,...,5}{% 
\pgfmathtruncatemacro\j{\c*\i}
\coordinate (\c;\j) at (60*\i:\c);  
} }
\foreach \i in {0,2,...,10}{% 
\pgfmathtruncatemacro\j{mod(\i+2,12)} 
\pgfmathtruncatemacro\k{\i+1}
\coordinate (2;\k) at ($(2;\i)!.5!(2;\j)$) ;}

\foreach \i in {0,3,...,15}{% 
\pgfmathtruncatemacro\j{mod(\i+3,18)} 
\pgfmathtruncatemacro\k{\i+1} 
\pgfmathtruncatemacro\l{\i+2}
\coordinate (3;\k) at ($(3;\i)!1/3!(3;\j)$)  ;
\coordinate (3;\l) at ($(3;\i)!2/3!(3;\j)$)  ;
 }

 %%%%%%%%% draw lines %%%%%%%%
 \foreach \i in {0,...,6}{% 
 \pgfmathtruncatemacro\k{\i}
 \pgfmathtruncatemacro\l{15-\i}
 \draw[thin,gray] (3;\k)--(3;\l);
 \pgfmathtruncatemacro\k{9-\i} 
 \pgfmathtruncatemacro\l{mod(12+\i,18)}   
 \draw[thin,gray] (3;\k)--(3;\l); 
 \pgfmathtruncatemacro\k{12-\i} 
 \pgfmathtruncatemacro\l{mod(15+\i,18)}   
 \draw[thin,gray] (3;\k)--(3;\l);}    
%%%%%%%%% some specific lines %%%%%%%%%% 
 \foreach \i in {0,2,...,10} {
   \pgfmathtruncatemacro\j{mod(\i+2,12)} 
   \draw[thick,dashed] (2;\i)--(2;\j);}     
%%%%%%%%% draw points %%%%%%%% 
\fill [gray] (0;0) circle (2pt);
 \foreach \c in {1,...,3}{%
 \pgfmathtruncatemacro\k{\c*6-1}    
 \foreach \i in {0,...,\k}{% 
   \fill [gray] (\c;\i) circle (2pt);}}  
%%%%%%%%% some specific points %%%%%%%%%%  
 \foreach \n in {0,3,...,15}{% 
   \draw (3;\n) circle (4pt);}
 \foreach \n in {1,3,...,11}{% 
   \draw (2;\n) circle (4pt);}
%%%%%%%%%% arrows %%%%%%%%%%%%
\draw[->,red,thick,shorten >=4pt,shorten <=2pt](0;0)--(2;3);
\draw[->,red,thick,shorten >=4pt,shorten <=2pt](0;0)--(2;1); 
\end{tikzpicture}  